    \documentclass{article}
    \usepackage{amsmath}
	\usepackage{amssymb}
    \usepackage{minted}
    \usepackage{xcolor} % Required for bgcolor in minted
    \definecolor{lightgray}{rgb}{0.9,0.9,0.9} % Define lightgray if not already defined
    \usepackage{fontspec} %fontspec is required for code font
    \setmonofont{JuliaMono}[Extension=.ttf, UprightFont=*-Regular, BoldFont=*-Bold, ItalicFont=*-RegularItalic, BoldItalicFont=*-BoldItalic, Contextuals=Alternate, Scale = 0.8]
    \usepackage{tabularx} % For tables that fit page width
    \usepackage{array} % For better column formatting
    \usepackage{booktabs} % For professional looking tables
	\usepackage{fullpage,graphicx,psfrag,amsfonts,verbatim, url}
    
    \title{\textbf{\textsf{Title: \texttt{sample\_code}}}}
	
	\author{John Doe}
    
    \begin{document}
    \maketitle
    
    
Some text and some \texttt{code}.

\begin{minted}[frame=lines,framerule=2pt,mathescape=true,bgcolor=lightgray,breaklines]{julia}
x = 1
# a comment
y = x+1
## this is another comment
\end{minted}


\begin{equation*}
\begin{aligned}
  a &= b + c \\
  & \quad + d + e \\
  f &= g - h
\end{aligned}
\end{equation*}



As shown in Equation, Euler's identity is a fundamental mathematical relationship.
\section{Einstein's Equation \texttt{E=m c\textasciicircum{}2}}
This is markdown. Consider: 
\begin{equation}\label{einstein}
E = mc^2
\end{equation}

Where Equation \eqref{einstein} demonstrates the relationship between energy \texttt{e}, mass \texttt{m}, and the speed of light \texttt{c}.

\begin{minted}[frame=lines,framerule=2pt,mathescape=true,bgcolor=lightgray,breaklines]{julia}
m = 1

c = 3e8

E = m*c^2
\end{minted}


Another multiline equation is as follows: 

\begin{equation}\label{Ax-eq-b}
\begin{aligned}
  Ax &= b \\
  x &\ge 0
\end{aligned}
\end{equation}
Here, Equation \eqref{Ax-eq-b} represents a system of linear equations with non-negativity constraints.

Here is a table:

\begin{tabular}{crr}
\toprule
Algorithm & $f(x^*)$ & Time (s) \\
\midrule
PDHG & 1.234 & 0.07 \\
B\&B & 1.230 & 3.12 \\
\bottomrule
\end{tabular}


Another table: 


\begin{tabularx}{\textwidth}{XXXXX}
\toprule
Field & Type & Mathematics & Implementation detail & Role in the package \\
\midrule
\texttt{\_id} & \texttt{Int} & Pure identifier; no direct math meaning & Incremented from global \texttt{NEXT\_ID[]} & Stable identity for hashing, comparisons, dictionary keys \\
\texttt{\_is\_leaf} & \texttt{Bool} & Indicates whether this is a fundamental vector in the Gram basis & \texttt{true} for leaf points, \texttt{false} for linear combinations & Determines Gram dimensioning and whether \texttt{.counter} is set \\
\texttt{decomposition\_dict} & \texttt{OrderedDict\{Point,Float64\}} & Stores coefficients of the linear form $X=\sum_i \alpha_i P_i$ & For a leaf, set to \texttt{\{self => 1.0\}}; for a composite, the sparse coefficient map & Drives conversion of inner products to linear forms over G \\
\texttt{counter} & \texttt{Union\{Int,Nothing\}} & Index i of the leaf in the Gram basis (only for leaves) & Set to \texttt{Point\_counter[]} at leaf creation, \texttt{nothing} otherwise & Used to size/index the Gram matrix and build evaluation vectors \\
\texttt{\_value} & \texttt{Union\{Vector\{Float64\},Nothing\}} & Numerical value of the vector after solving the PEP & \texttt{nothing} until \texttt{solve!} writes results back & Enables \texttt{eval} to return a concrete vector after solve \\
\bottomrule
\end{tabularx}



\end{document}

\documentclass{article}
\usepackage{amsmath}
\usepackage{minted}
\usepackage{xcolor} % Required for bgcolor in minted
\definecolor{lightgray}{rgb}{0.9,0.9,0.9} % Define lightgray if not already defined
\usepackage{fontspec} %fontspec is required for code font
\setmonofont{JuliaMono}[Extension=.ttf, UprightFont=*-Regular, BoldFont=*-Bold, ItalicFont=*-RegularItalic, BoldItalicFont=*-BoldItalic, Contextuals=Alternate, Scale = 0.8]

\title{Title: \texttt{sample code}}

\begin{document}
\maketitle


Some text 

\begin{minted}[frame=lines,framerule=2pt,mathescape=true,bgcolor=lightgray,breaklines]{julia}
x = 1
# a comment
y = x+1
## this is another comment
\end{minted}


\begin{equation*}
\begin{aligned}
  a &= b + c \\
  & \quad + d + e \\
  f &= g - h
\end{aligned}
\end{equation*}



As shown in Equation, Euler's identity is a fundamental mathematical relationship.
\section{Einstein's Equation \texttt{E=m c\textasciicircum{}2}}
This is markdown. Consider: 
\begin{equation}\label{einstein}
E = mc^2
\end{equation}

Where Equation \eqref{einstein} demonstrates the relationship between energy \texttt{e}, mass \texttt{m}, and the speed of light \texttt{c}.

\begin{minted}[frame=lines,framerule=2pt,mathescape=true,bgcolor=lightgray,breaklines]{julia}
m = 1

c = 3e8

E = m*c^2
\end{minted}


Another multiline equation is as follows: 

\begin{equation}\label{Ax-eq-b}
\begin{aligned}
  Ax &= b \\
  x &\ge 0
\end{aligned}
\end{equation}
Here, Equation \eqref{Ax-eq-b} represents a system of linear equations with non-negativity constraints.


\end{document}
